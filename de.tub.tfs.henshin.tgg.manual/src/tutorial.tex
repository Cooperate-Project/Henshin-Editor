\chapter{Installation}\label{chap:Installation}
\index{Installtion}

\section{Installationsanleitung}
Umm den Henshin TGG Editor nutzen können ist vorerst die Installation von Eclipse und einigen Plugins notwendig, auf denen der Henshin TGG Editor basiert.

Um den Henshin TGG Editor möglichst fehlerfrei nutzen zu können wird empfohlen von der Downloadseite des Eclipseprojekts\footnote{http://www.eclipse.org/downloads/} die "Eclipse IDE for Java Developers" in der Version 3.7\footnote{Version 3.7 entspricht der Eclipse Indigo Version} herunterzuladen und zu installieren.

Nach der Installation installiert man über den Eclipse Marketplace folgende Plugins herunter: Ecore Tools SDK (Incrubation), EMF - Eclipse Modeling Framework SDK, Graphical Editing Framework GEF SDK, Graphical Modeling Framework (GMF) Runtime SDK. Sind diese Plugins installiert sind, ist es notwendig das Muvitor-Plugin der TU Berlin, die Plugins EMF-Henshin-Interpreter, EMF-Henshin-Matching, EMF-Henshin-Model und zuletzt den TGG-Editor selbst zu installieren. Das geht ganz einfach mit dem ausgelieferten TGG-Editor.zip Archiv. Dieses wird komplett in den Pluginordner der Eclipseinstallation entpackt. Nach einem Neustart stehen alle Plugins zur Verfügung.

\section{Erster Schritt}
Bevor man mit dem Editieren einer Tripelgraphgrammatik beginnen kann ist es notwendig die Metamodelle (siehe \ref{subsec:Typgraph}) für Source-, Correspondence- und Targetteilgraph der Tripelgraphgrammatik zu erstellen und zu importieren.
Die Metamodelle werden als Ecoremodell mit dem EMF Ecore Baum-Editor oder dem graphischen Ecore-Editor von GMF in Eclipse erstellt.
Sind die Metamodelle fertig kann man wie im Beispielworkflow (siehe Kapitel \ref{chap:Workflow}) seine eigene Tripelgraphgrammatik erzeugen.
